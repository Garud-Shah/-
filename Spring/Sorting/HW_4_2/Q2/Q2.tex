\documentclass{article}
\usepackage{garudstyle}
\title{
    \begin{center}
        Programming 3 - WVA 2023-24
    \end{center}
    \begin{center}
        \textbf{4/2 Homework - Question 2}
    \end{center}}
\date{April 9th, 2024}
\author{Garud Shah}
\begin{document}
\maketitle
\section{Time Complexity}
This has a time complexity of $O(N^2)$: the worst case scenario - which is when the list is in decending order, since the $k$th largest 
element will only "bubble" up and become sorted on the $k$th cycle. This is, since the number of comparisons on the $i$th cycle is $N-1-i$:
\begin{align}
    & \sum_{i=1}^{N-1} N-1-i \\
    &= \sum_{i=1}^{N-1} i \\
    &= \dfrac{N \cdot (N - 1)}{2} \\
    &= \dfrac{N^2}{2} - \dfrac{N}{2} \\
    &= \boxed{O(n^2)},
\end{align}
completing the proof this algorithim is at least $O(n^2)$. \newline
For the proof it works, which is needed to prove that our sum is valid, after $N - k - 1$ cycles it will have sorted everything as the 
$k$th largest element, which will have position $N - 1 - k$ in a $0$-indexed list. Then forces the smallest element to the start or index $0$ 
as all other indexes will have been taken by larger elements in sorted order, so we can safely end here. This completes the proof that the 
algorithim works and that our sum is valid.
\section{Space Complexity}
As only single comparisons and swaps are done at a time, and since the algorithim works without modification, it has $O(1)$ space complexity.
\section{Algorithim Grades}
Note: this is question 3.
\end{document}